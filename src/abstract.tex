In recent years, Convolutional Neural Network (CNN) has been widely used in robotics, which has dramatically improved the perception and decision-making ability of robots.
In order to implement energy-efficient CNN on embedded systems, a series of CNN accelerators have been designed. However, despite the high energy efficiency on CNN accelerators, it is difficult for robotics developers to use it. Since the various functions on the robot are usually implemented independently by different developers, simultaneous access to the CNN accelerator by these multiple independent processes will result in hardware resource conflicts.

To handle the above problem, we propose an INterruptible CNN Accelerator (INCA) to enable multi-tasking on CNN accelerators. In INCA, we propose a Virtual-Instruction-based interrupt method (VI method) to support multi-task on CNN accelerators. Based on INCA, we deploy the Distributed Simultaneously Localization and Mapping (DSLAM) on an embedded FPGA platform. We use CNN to implement two key components in DSLAM, Feature-point Extraction (FE) and Place Recognition (PR), so that they can both be accelerated on the same CNN accelerator. Experimental results show that, compared to the layer-by-layer interrupt method, our VI method reduces the interrupt responding latency to 1\%.